
\section{Ausrichtung des Solarpanels}\label{sec:ausrichtung des Solarpanels}
Um die Leistungsaufnahme des Solarpanels zu optimieren ist es notwendig dieses direkt auf die Sonne auszurichten und diese Ausrichtung auch in geeigneten Zeitabständen zu korrigieren. Im Vergleich mit einem fest ausgerichteten Solarpanel konnten J. Rizek \emph{et al.} mit einem nachgeführten Solarpanel beispielsweise die Leistungsaufnahme um durchschnittlich 30\% erhöhen~\cite{Rizek2008}.

Hierfür kommen grundsätzlich verschiedene Methoden in Frage. In diesen Fall soll die Position der Sonne relativ zur Wetterstation auf Grundlage des Längen- und Breitengrades, der Uhrzeit und des Kalendertages berechnet werden. Diese Information werden über das GPS-Modul bereitgestellt. Anschließend wird das Solarpanel mit Hilfe der Motoren, des Kompass-Moduls und des, am Panel befestigten Neigungssensors, auf die Sonne ausgerichtet.

Im folgenden wird zunächst die Berechnung der Sonnenposition und anschließend die Positionierung des Panels beschrieben.

\subsection{Berechnung der Sonnenposition}\label{sec:berechnung_der_sonnenposition}



%%% Local Variables:
%%% mode: latex
%%% TeX-master: "../termpaper"
%%% End:
